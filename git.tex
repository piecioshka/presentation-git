\documentclass{beamer}

\mode<presentation>
{
  \usetheme{PaloAlto}
  \usecolortheme{default}
  \usefonttheme{default}
  \setbeamertemplate{navigation symbols}{}
  \setbeamertemplate{caption}[numbered]
}

\usepackage{polski}
\usepackage[utf8x]{inputenc}
\usepackage{hyperref}
\usepackage{color}
\usepackage{pgfpages}
\pgfpagesuselayout{resize to}[physical paper width=8in, physical paper height=6in]

\title[Git]{Git - system kontroli wersji}
\author{Piotr Kowalski}
\institute{Wyższa Szkoła Informatyki Stosowanej i Zarządzania}
\date{24-05-2013}

% ------------------------------------------------------------------------------

\begin{document}
	
% --------------------------------------

\begin{frame}
  \titlepage
\end{frame}

% --------------------------------------

\section{Wprowadzenie}

\begin{frame}{Wprowadzenie}
\begin{itemize}
  \item Co to jest system kontroli wersji?
  \item Dlaczego \texttt{git} jest zdobył serca programistów?
\end{itemize}
\vskip 1cm
\begin{block}{System kontroli wersji}
Oprogramowanie służące do śledzenia zmian głównie w kodzie źródłowym oraz pomocy programistom w łączeniu zmian dokonanych przez wiele osób w różnych momentach.
\end{block}
\end{frame}

% --------------------------------------

\section{Rodzina systemów}

\begin{frame}{Scentralizowane}
\begin{itemize}
	\item RCS
	\item CVS
	\item Subversion
	\item GNU CSSC, klon SCCS
	\item JEDI VCS
\end{itemize}
\end{frame}

\begin{frame}{Rozproszone}
\begin{itemize}
	\item Bazaar
	\item Codeville
	\item Darcs
	\item Git
	\item GNU Arch
	\item Mercurial
	\item Monotone
	\item svk
\end{itemize}
\end{frame}

\begin{frame}{Zamknięte (własnościowe) systemy kontroli wersji}
\begin{itemize}
	\item BitKeeper firmy BitMover
	\item Code Co-op firmy Reliable Software
	\item Perforce firmy Perforce Software
	\item Rational ClearCase firmy IBM
	\item Sablime firmy Lucent Technologies
	\item StarTeam firmy Borland
	\item Visual SourceSafe firmy Microsoft
	\item Visual Studio Team Foundation Server firmy Microsoft
\end{itemize}
\end{frame}

\begin{frame}{Historia}
\end{frame}

\begin{frame}{Cele}
\end{frame}

% --------------------------------------

\section{Składowe}

\begin{frame}{Branch}
\end{frame}

\begin{frame}{Tag}
\end{frame}

% --------------------------------------

\section{Git w życiu codziennym}

\begin{frame}{Praca lokalna}
\end{frame}

\begin{frame}{Praca zdalna}
\end{frame}

% --------------------------------------

\section{Podsumowanie}

\begin{frame}{Podsumowanie}
\end{frame}

\begin{frame}{Linki}
\begin{itemize}
  \item \url{http://git-scm.com/book/pl/Pierwsze-kroki-Wprowadzenie-do-kontroli-wersji}
  \item \url{https://github.com/piecioshka/git-presentation}
\end{itemize}
\end{frame}

\end{document}
